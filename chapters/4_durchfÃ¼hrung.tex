\chapter{Versuchsdurchführung}
Vor Versuchsbeginn mussten einige Kalibrierungen durchgeführt werden. Bedingt durch den Aufbau liegt der Torsionsdraht
dicht an den kurzen Spulen. Um die Messungen nicht durch Reibeffekte zu verzerren musste dafür Sorge getragen werden,
dass der Torsionsdraht nicht in Kontakt mit den kurzen Spulen steht. Nachdem ein Spulenstrom von $I = \SI{0}{A}$
sichergestellt wurde musste der Torsionsdraht so justiert werden, dass der Laser die Skala bei dem Wert $0$ trifft.
\par\medskip
\hspace{1cm}Der Spulenstrom selbst wurde durch eine im Zehntelbereich variable Konstantstromquelle eingestellt.
Da die Ausschläge sich bei einem Gleichgewicht zwischen Rückstell- und magnetischer Kraft einstellen und die das System
Stabmagnet-Spulenfeld-Torsionsdraht kaum gedämpft sind wurde die Auslenkung durch einen einige Millimeter in Mineralöl
getauchte Stift so gedämpft, dass sich das System in akzeptabler Zeit einpendelt.
\par
\hspace{1cm}Im Falle der langen Spule mit der Windungszahl $n=240$ war der größte Ausschlag zu erwarten. Vor Beginn der ersten Messung
wurde also überprüft, bei welchem Spulenstrom der Ausschlag sein mechanisches Maximum erreicht und darüber der mögliche
Messbereich ermittelt.
\par
\hspace{1cm}Der mechanisch maximal mögliche Ausschlag lag bei $\SI{44,4}{Skt}$. Dieser wurde bei einem Spulenstrom von $I=\SI{0,78}{A}$
für $N=240$ bzw. $I=\SI{1,58}{A}$ für $N=120$ erreicht. Hieraus ergaben sich für die Messung des Ausschlags bei der Langen
Spule eine Indexweite von $I = \SI{0,1}{A}$ (vgl. \tabelle{tab:mess1}). Für die übrigen Messungen wurde ein maximaler
Spulenstrom von $I=\SI{2}{A}$ vorgeschlagen wobei die Ausschläge der kurzen Spule mit $D=\SI{0,4}{m}$ so klein wurden, dass
eine Schrittweite unter $I=\SI{0,4}{A}$ zu Auslenkungen in der Umgebung der Messunsicherheit geführt hätten (vgl. \tabelle{tab:mess2}
und \tabelle{tab:mess3}).