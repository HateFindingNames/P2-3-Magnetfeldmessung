\chapter{Fazit}
Der Versuch konnte ein tieferes Verständniss über die Zusammenhänge zwischen Spulengeometrien und den resultierenden
Feldstärken vermitteln. Interessant war auch, zu sehen wie schnell sich eine scheinbar geringe Änderung des Spulenstromes
auf die Feldstärke auswirkt.
\par
\hspace{1cm}Zu vermissen war eine - wenn auch für den konkreten Versuch nicht notwendige - Möglichkeit zur Kalibrierung
des Rückstellmomentes der Torsionsfeder. Weiter ist ein einstellbarer Diodenstrom des Lasers oder ein dunklerer Grund der
Skala wünschenswert. Da der Laser im Istzustand einen relativ großen und hellen Leuchtpunkt erzeugt, erhöht er so unnötig
den Messfehler der Auslenkung und trägt darüber hinaus auch zur Ermüdung der Augen des Experimentators bei.